\documentclass[a4paper,12pt]{report}
\usepackage[english]{babel}
\usepackage[T1]{fontenc}
\usepackage[utf8]{inputenc}
\usepackage{lmodern}
\usepackage{microtype}

\usepackage{hyperref}

\title{Bibliographical Project \\ [1ex]\large X-Rays and Radiography \\ (based on the work of Wilhem Röntgen)  }
\author{    }

\date{October 1, 2023}

\begin{document}
\maketitle 
\tableofcontents
\newpage
\section*{Introduction}
In the context of our \textit{Vibrations and Waves} class we were assigned to produce a bibliographical, scientific and technical report on a subject related to our lecture and interest int the subject. This project has to be done by group. \\

As engineering student in the beginning of our cursus it is quite challenging to bu(x, y)ild from scratch such an intellectual work but we are more than motivated to achieve an objective like this one. \\
\
An important point is to find the appropriate framework of study in order to determine what we want to dig deeper or not and to avoid unnecessary informations.
\addcontentsline{toc}{part}{Introduction}
\part{Preliminary Researches}
\chapter{Historical context}
\section{Physics state of the art}
\subsection{Main discoveries}

In the late 19th century, during the time of Wilhelm Röntgen, the field of physics was marked by a profound sense of curiosity and exploration. Scientists of the era were on the cusp of major discoveries, delving into the fundamental nature of the physical world. Röntgen's groundbreaking work with X-rays, which he discovered in 1895, exemplified this spirit of inquiry. At the time, the understanding of electromagnetic radiation was rapidly evolving, and Röntgen's accidental discovery of X-rays revolutionized not only the field of physics but also medicine and numerous other scientific disciplines. This period was characterized by a dynamic interplay between theoretical exploration and experimental innovation, laying the foundation for the rapid advancements that would shape the 20th century's scientific landscape.
\subsection{Main physicists}
During Wilhelm Röntgen's time in the late 19th century and early 20th century, several prominent physicists made significant contributions to the field of physics. Some of the main physicists of that era include:
\begin{enumerate}


    \item Albert Einstein (1879-1955): Einstein is best known for his theory of relativity, which includes the famous equation E=mc². His work in 1905 on the photoelectric effect was crucial for the development of quantum mechanics.

\item Max Planck (1858-1947): Planck is considered the father of quantum theory. He introduced the concept of quantization of energy and formulated Planck's law, which laid the foundation for modern quantum physics.

\item Niels Bohr (1885-1962):** Bohr made significant contributions to our understanding of the atomic structure. He developed the Bohr model of the atom, which explained the quantization of electron orbits.

\item Marie Curie (1867-1934):** Curie conducted pioneering research on radioactivity and was the first woman to win a Nobel Prize. Her work was fundamental in understanding the properties of radioactive materials.

\item Ernest Rutherford (1871-1937):** Rutherford conducted the famous gold foil experiment, leading to the discovery that atoms have a nucleus. His work on the atomic model was groundbreaking.

\item Maxwell Planck (1853-1942):** Known for his work in statistical mechanics and thermodynamics, Planck's contributions helped bridge the classical and quantum physics realms.

\item J.J. Thomson (1856-1940):** Thomson discovered the electron and made significant contributions to the understanding of atomic and subatomic particles.

\item Werner Heisenberg (1901-1976):** Heisenberg formulated the famous Heisenberg Uncertainty Principle, a fundamental concept in quantum mechanics.
\end{enumerate}

These physicists, along with many others, collectively advanced our understanding of the physical world during the time when Wilhelm Röntgen's discovery of X-rays was made. Their work laid the groundwork for the modern era of physics and had a profound impact on science and technology.
\subsection{Wilhem Röntgen}
Wilhelm Conrad Röntgen was a German physicist who is best known for his discovery of X-rays. He was born on March 27, 1845, in Lennep, Prussia (now Remscheid, Germany), and he passed away on February 10, 1923, in Munich, Germany.

Röntgen's groundbreaking discovery of X-rays occurred in 1895 while he was conducting experiments with cathode rays (electrons) in a vacuum tube. He noticed that a screen coated with a fluorescent material in his laboratory began to glow even though it was not in the direct path of the cathode rays. He realized that this mysterious radiation was capable of passing through solid objects and producing images on photographic plates.

Röntgen's work led to the development of X-ray photography, which revolutionized medicine and various scientific fields. X-rays became a crucial tool for medical diagnosis, allowing doctors to visualize the internal structures of the human body non-invasively. For his discovery of X-rays, Röntgen was awarded the first Nobel Prize in Physics in 1901.

Wilhelm Röntgen's contribution to science and medicine had a profound and lasting impact, and X-ray technology continues to be widely used for diagnostic and research purposes to this day.
\chapter{Work of Wilhem Röntgen}


This documents is a preliminary work for our Bibliographical group project on X-Rays and Radiography (Work of Wilhem Röntgen). 

Also I am learning the \LaTeX\ software so here is a test paragraph.
\end{document}
